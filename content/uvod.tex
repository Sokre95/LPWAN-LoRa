\chapter{Uvod}
\label{chapter:uvod}
Gotovo uvijek u realnom svijetu imamo slučaj da su resursi ograničeni. U svijetu elektroničkih komunikacija resurse predstavljaju električna energija, a u bežičnim komunikacijama još je bitan i faktor dostupnosti radio-frekvencijskog spektra. Napretkom tehnologije, na svijetu se generira sve veća količina podataka te raste potreba za prijenosom tih podataka, a još k tome raste i broj sudionika u komunikacijskim mrežama. Sve te činjenice guraju dalje razvoj komunikacijskih tehnologija, a prije svega bežičnih odnosno mobilnih tehnologija pred koje se stavljaju sve veći izazovi.
\newline

Kada govorimo o bežičnim komunikacijima prvo nam na pamet padaju mobilne mreže 3. i 4. generacije (3G i LTE) ili WiFi (IEEE 802.11 a/b/g/n/ac) bežična tehnologija koja se koristi unutar LAN mreža. Takve tehnologije omogučavaju prijenos velike količine podataka, za današnje pojmove velikom brzinom. Nedostatak ovih tehnologija i uređaja koji koriste ove tehnologije je relativno velika potrošnja električne energije. To je između ostalog jedan od razloga zašto današnje mobilne uređaje, pametne telefone i ostale baterijski napajane uređaje svakog dana moramo spajati na punjenje. Usmjerivači i slični mrežni uređaji ne mijenjaju položaj te stoga pristup električnoj energiji za takve primjere uređaja ne predstavlja problem.
\newline

Međutim, postoje uređaji koji moraju mjesecima ili čak godinama raditi bez mogučnosti trajnog iskorištavanja električne energije iz gradske mreže ili svakodnevnog punjenja. Logičan način za napajanje takvih uređaja je baterija i/ili manji solarni panel ili sličan način prikupljanja energije iz okoline odnosno iskorištavanja obnovljivih izvora energije. Takvi baterijski napajani uređaji, kako bi dugo mogli obavljati svoju funkciju, moraju trošiti iznimno malo električne energije, a istovremeno imaju potrebu za bežičnom komunikacijom koja redovno troši najviše energije. Uređaji s takvim potrebama zahtjevaju specifičnu komunikacijsku tehnologiju i protokol.
\newline

LPWAN (Low-Power Wide-Area Network) je bežična komunikacijska tehnologija koja je dizajnirana da omoguči komunikaciju na velike udaljenosti, uz malu potrošnju električne energije, između uređaja koji generiraju manje količine podataka (npr. senzori koji rijetko mjere spore promjene fizikalne veličine koja se mjeri). Jedna od najzastupljenijih LPWAN tehnologija je LoRa koja će detaljnije biti opisana u poglavlju \ref{chapter:lora}.